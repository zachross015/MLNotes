\documentclass{book}[a5paper]
\usepackage[a5paper]{geometry}
\usepackage[utf8]{inputenc}
\usepackage{amsmath}
\usepackage{amsthm}
\usepackage{amssymb}
\usepackage[font=small,labelfont=bf]{caption}
\usepackage{wrapfig}
\usepackage{enumitem}
\usepackage{import}

\usepackage{xfp}
\usepackage{pgfplots, pgfplotstable}
\usetikzlibrary{pgfplots.units}
\pgfplotsset{compat=1.16}

\usepackage{tikz}
\usepackage{float}

% Speed up the compile time for TIKZ
%\usepgfplotslibrary{external}
%\tikzexternalize

\newcommand{\placeholder}{[Insert Section Here]}

\newtheorem{theorem}{Theorem}[section]
\newtheorem{corollary}{Corollary}[theorem]
\newtheorem{lemma}[theorem]{Lemma}
\newtheorem*{remark}{Remark}


% Exercise section
\newlist{exercise}{enumerate}{5}
\setlist[exercise]{label*=\arabic*.,ref=\arabic*,before={\subsection*{Exercises}}}
\let\ex\item

% Style to select only points from #1 to #2 (inclusive)
\pgfplotsset{select coords between index/.style 2 args={
    x filter/.code={
        \ifnum\coordindex<#1\def\pgfmathresult{}\fi
        \ifnum\coordindex>#2\def\pgfmathresult{}\fi
    }
}}

\title{The Art of Machine Learning}
\author{Zachary Ross}
\date{November 2019}
