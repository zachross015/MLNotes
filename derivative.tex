\subsection{Finding extrema}

It is often desirable to have the \emph{maxima} and \emph{minima}, otherwise known as \emph{extrema}, in a function's range. For example, when throwing a ball in the air, it is good to know what time it is highest in the air and what that height is. With added complexity, we can use these extrema to optimize operations we are performing on a data set, or maximizing revenue in a storefront, etc.

% TODO: Include a brief history of derivatives and calculus. Maybe the origin of the term derivative

\emph{The derivative of a function is its rate of change over an infinitely small interval.} This interval is narrowed to a singularity so that it is almost considered a pointwise change. When describing the derivative of a function $f(x)$ with respect to a variable $x$, we notate this as
\begin{equation*}
    \dx f(x) \quad \text{or} \quad \dx[f] \quad \text{or} \quad f'(x)
\end{equation*}, which put simply is the change in $f(x)$ over the change in $x$ and is, again, minimized so that the change in $x$ is infinitely small (nearly 0).


\begin{psec}{Constants}

    \begin{equation}\label{eq:derivative constant}
        \dx c = 0
    \end{equation}

\end{psec}

\begin{psec}{Power rule}

    \begin{equation}\label{eq:derivative power}
        \dx x^c = c x^{(c - 1)}
    \end{equation}

\end{psec}

% TODO: Cover polynomials in here
\begin{psec}{Addition and subtraction}

    \begin{equation}\label{eq:derivative add}
        \dx (f(x) + g(x)) = \dx[f] + \dx[g]
    \end{equation}

\end{psec}

\begin{psec}{Product rule}

    \begin{equation}\label{eq:derivative prod}
        \dx \left(f(x)*g(x)\right) = \dx[f] *  g(x) + f(x) *
        \dx[g]
    \end{equation}

\end{psec}

\begin{psec}{Chain rule}

    \begin{equation}\label{eq:derivative chain}
        \dx (f \circ g)(x) = \diff{f}{g} * \diff{g}{x}
    \end{equation}

\end{psec}

\begin{psec}{Quotient Rule} The quotient rule is just an extension of the
    power, product, and chain rules.
    \begin{equation}
        \dx \left(\frac{f(x)}{g(x)}\right) = \frac{\dx[f] * g(x) - f(x) * \dx[g]
        }{g(x)^2}
    \end{equation}.

\end{psec}

\begin{psec}{Exponent rule}

    \begin{equation}
        \dx c ^ {f(x)} = c ^ {f(x)} * \dx[f]
    \end{equation}

\end{psec}

\begin{psec}{Logarithm rule}

    \begin{equation}
        \dx \log_c f(x) = \frac{1}{\ln c} * \frac{1}{f(x)} * \dx[f]
    \end{equation}

\end{psec}


\begin{psec}{Trigonometric rules}


\end{psec}
