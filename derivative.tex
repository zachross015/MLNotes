\subsection{Derivatives}

Let $f(x)$ be any continuous function and let $a, b$ be two points in the domain of $f$ with $a\leq b$. The \emph{rate of change in $f$ from $a$ to $b$ is the speed that $f$ changes over the interval $[a,b]$} and is measured by the slope of the line between
the points $(a, f(a))$ and $(b, f(b))$
\begin{equation}\label{eq:pre roc}
    \frac{f(b)-f(a)}{b-a}.
\end{equation}
This is also the line running tangent to these points.
As an example of this, let $f(x)=x^2$. Our rate of change for $a=0$ and $b=1$ would then be 1, and if we instead set $a=1$ and $b=2$, the rate of change comes out as 3. By minimizing the distance between $a$ and $b$, we get a better estimation for our rate of change, since narrowing this margin reduces any smoothing that occurs in bigger margins. For an idea of this, take $a=-2$ and $b=2$. Then our rate of change is $0$, which does not seem accurate since at $f(a)$ we are reducing the speed which we move down and so we would expect our rate of change to be negative. If we instead set $b=0$, we have a rate of change of -2, which seems more accurate considering the graph of $x^2$ bottoms out when $x=0$. As such, the negative value for $f(a)$'s rate of change tells us that the \emph{rate that $f(x)$ decreases between -2 and 0 is always decreasing}.

Keeping this in mind, we instead want to examine the affect of closing the margin $b-a$, and so it is much more convenient to change the notation to reflect this. Let $x_1 = a$ and let $h = b - a$. Equation (\ref{eq:pre roc}) can then be simplified to
\begin{equation}\label{eq:roc}
    \frac{f(x_1+h)-f(x_1)}{h}.
\end{equation}.

We then get the \emph{instantaneous rate of change of $f$ at $x_1$} by
narrowing the offset $h$ in Equation (\ref{eq:roc}) to an infinitely small
amount, nearly 0, at $(x_1, f(x_1))$.
\begin{equation}\label{eq:iroc}
    \dx[f] (x_1) = \lim_{h\rightarrow 0} \frac{f(x_1 + h) - f(x_1)}{h} \quad .
\end{equation}
This is otherwise known as the slope of the line that runs tangent to the \emph{point} $(x_1,f(x_1))$. The instantaneous rate of change can be thought of as finding the direction that $f$ is moving in at the exact point $x_1$.

By extrapolating Equation (\ref{eq:iroc}) to the rest of $f$'s domain, we define \emph{the derivative of $f$ with respect to $x$ as the
instantaneous rate of change for all $x$ in the domain of $f$}
\begin{equation}\label{eq:derivative}
    \dx[f] (x) = \lim_{h\rightarrow 0} \frac{f(x + h) - f(x)}{h} \quad .
\end{equation}

We will use a couple of alternate notations for the derivative
throughout the text
\begin{equation*}
    \dx[f](x) \quad \text{or} \quad \dx f(x) \quad \text{or} \quad f'(x)
\end{equation*}, all of which represent the derivative of a function $f$ with respect to its input, $x$.


% TODO: Add more here
% Include specifications on the variable with which the differentiation is in
% respect of i.e. dx or dy or what not

We will now define the nine rules by which most differentiation can be performed. Let $c_1$ and $c_2$ be constants, let $x$ and $y$ be variables, and let $f(x)$ and $g(x)$ both be continuous and real valued functions.

\begin{psec}{Constants}\label{rule:derivative constants} For constants,
    we have
    \begin{equation}\label{eq:derivative constant}
        \dx c_1 = 0
    \end{equation}
    . This is an intuitive consequence since a constant by definition does not
    change, and thus has no rate of change.

    If we differentiate a variable that does not change with respect to changes in our differentiating variable, then its derivative is treated as a constant. The same applies to a function. So the variable $y$ or $h(y)$ would, with respect to $x$, both have the derivatives
    \begin{equation*}
        \dx y = 0, \quad \quad \dx h(y) = 0
    \end{equation*} if and only if the variable $y$ is independent of $x$. This is once again an intuitive consequence since the variables will remain the same when $x$ is changed.

\end{psec}

\begin{psec}{Power rule}\label{rule:derivative powers} For any constant $c$ and
    variable $x$, we have
    \begin{equation}\label{eq:derivative power}
        \dx x^{c_1} = c_1 x^{(c_1 - 1)}
    \end{equation}
    . To understand this consider the case when $c_1=3$ so that $f(x) = x^3$. We have
    \begin{align*}
        \dx[f] &= \lim_{h\rightarrow 0} \frac{(x + h)^3 - x^3}{h} & &
        \text{\footnotesize by definition (\ref{eq:derivative})}\\
               &= \lim_{h\rightarrow 0} \frac{x^3 +3x^2h + 3xh^2 + h^3 -
               x^3}{h} & & \text{\footnotesize by expanding $(x + h)^3$}\\
               &= \lim_{h\rightarrow 0} 3x^2 + 3xh + h^2 & &
               \text{\footnotesize by cancellation}\\
               &= \lim_{h\rightarrow 0} 3x^2 & & \text{\footnotesize since $h\rightarrow 0$} \\
               &= 3x^2 & \text{}
    \end{align*}.

\end{psec}

% TODO: Cover polynomials in here
\begin{psec}{Addition and subtraction}\label{rule:derivative add}

    \begin{equation}\label{eq:derivative add}
        \dx (f(x) + g(x)) = \dx[f](x) + \dx[g](x)
    \end{equation}

\end{psec}

\begin{psec}{Product rule}\label{rule:derivative product}

    \begin{equation}\label{eq:derivative prod}
        \dx \left(f(x)*g(x)\right) = \dx[f](x) *  g(x) + f(x) *
        \dx[g](x)
    \end{equation}

\end{psec}

\begin{psec}{Chain rule}\label{rule:derivative chain}

    \begin{equation}\label{eq:derivative chain}
        \dx (f \circ g)(x) = \diff{f}{g} * \diff{g}{x}(x)
    \end{equation}

\end{psec}

\begin{psec}{Quotient Rule}\label{rule:derivative quotient} The quotient rule is just an extension of the
    power, product, and chain rules.
    \begin{equation}
        \dx \left(\frac{f(x)}{g(x)}\right) = \frac{\dx[f](x) * g(x) - f(x) * \dx[g](x)
        }{g(x)^2}
    \end{equation}.

\end{psec}

\begin{psec}{Exponent rule}\label{rule:derivative exponent}

    \begin{equation}
        \dx \ln(f(x)) * c ^ {f(x)} = c ^ {f(x)} * \dx[f](x)
    \end{equation}

\end{psec}

\begin{psec}{Logarithm rule}\label{rule:derivative log}

    \begin{equation}
        \dx \log_c f(x) = \frac{1}{\ln c} * \frac{1}{f(x)} * \dx[f] (x)
    \end{equation}

    For the natural log, this is simplified to
    \begin{equation*}
        \dx \ln f(x) = \frac{1}{f(x)} * \dx[f].
    \end{equation*}

\end{psec}


\begin{psec}{Trigonometric rules}


\end{psec}
