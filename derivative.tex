\subsection{Derivatives}

Let $f(x)$ be any continuous function. By finding the slope of the line between
the point $f(x)$ and some point a little further $f(x+h)$, we come up with the
\emph{rate of change} for the interval from $x$ to $x + h$ using the slope
formula
\begin{equation}\label{eq:roc}
    \frac{f(x+h)-f(x)}{h}
\end{equation}
. We find the \emph{instantaneous rate of change of $f$ at $x$} by
narrowing the offset $h$ in Equation \ref{eq:roc} to an infinitely small
amount. By extrapolating this idea to the rest of the domain that $x$
belongs to, we \emph{define the derivative of $f$ with respect to $x$ as the
rate of change for all $x$ in the domain of $f$.}
\begin{equation}\label{eq:derivative}
    \dx[f] = \lim_{h\rightarrow 0} \frac{f(x + h) - f(x)}{h}
\end{equation}.

We will use a couple of alternate notations for the derivative
throughout the text
\begin{equation*}
    \dx f(x) \quad \text{or} \quad f'(x) \quad \text{or} \quad \dx[f]
\end{equation*}.


% TODO: Add more here
% Include specifications on the variable with which the differentiation is in
% respect of i.e. dx or dy or what not


\begin{psec}{Constants}\label{rule:derivative constants} For any constant $c$,
    we have
    \begin{equation}\label{eq:derivative constant}
        \dx c = 0
    \end{equation}
    . This is an intuitive consequence since a constant by definition does not
    change, and thus has no rate of change.

\end{psec}

\begin{psec}{Power rule}\label{rule:derivative powers} For any constant $c$ and
    variable $x$, we have
    \begin{equation}\label{eq:derivative power}
        \dx x^c = c x^{(c - 1)}
    \end{equation}
    . To understand this consider the case when $c=3$ so that $f(x) = x^3$. We have 
    \begin{align*}
        \dx[f] &= \lim_{h\rightarrow 0} \frac{(x + h)^3 - x^3}{h} & &
        \text{\footnotesize by definition (\ref{eq:derivative})}\\
               &= \lim_{h\rightarrow 0} \frac{x^3 +3x^2h + 3xh^2 + h^3 -
               x^3}{h} & & \text{\footnotesize expanding $(x + h)^3$}\\
               &= \lim_{h\rightarrow 0} 3x^2 + 3xh + h^2 & &
               \text{\footnotesize cancellation}\\
               &= \lim_{h\rightarrow 0} 3x^2 & & \text{\footnotesize $h\rightarrow 0$} \\
               &= 3x^2 & \text{}
    \end{align*}.
    
    Another case is when $c = \frac{1}{2}$ so that $f(x) = (x)^{\frac{1}{2}}$ or
    $\sqrt{x}$. By this rule, we end up with 
    \begin{equation*}
        \dx[f] = \recip{2} * (x)^{-\recip{2}} = \recip{2\sqrt{x}}
    \end{equation*}

\end{psec}

% TODO: Cover polynomials in here
\begin{psec}{Addition and subtraction}\label{rule:derivative add}

    \begin{equation}\label{eq:derivative add}
        \dx (f(x) + g(x)) = \dx[f] + \dx[g]
    \end{equation}

\end{psec}

\begin{psec}{Product rule}\label{rule:derivative product}

    \begin{equation}\label{eq:derivative prod}
        \dx \left(f(x)*g(x)\right) = \dx[f] *  g(x) + f(x) *
        \dx[g]
    \end{equation}

\end{psec}

\begin{psec}{Chain rule}\label{rule:derivative chain}

    \begin{equation}\label{eq:derivative chain}
        \dx (f \circ g)(x) = \diff{f}{g} * \diff{g}{x}
    \end{equation}

\end{psec}

\begin{psec}{Quotient Rule}\label{rule:derivative quotient} The quotient rule is just an extension of the
    power, product, and chain rules.
    \begin{equation}
        \dx \left(\frac{f(x)}{g(x)}\right) = \frac{\dx[f] * g(x) - f(x) * \dx[g]
        }{g(x)^2}
    \end{equation}.

\end{psec}

\begin{psec}{Exponent rule}\label{rule:derivative exponent}

    \begin{equation}
        \dx c ^ {f(x)} = c ^ {f(x)} * \dx[f]
    \end{equation}

\end{psec}

\begin{psec}{Logarithm rule}\label{rule:derivative log}

    \begin{equation}
        \dx \log_c f(x) = \frac{1}{\ln c} * \frac{1}{f(x)} * \dx[f]
    \end{equation}

\end{psec}


\begin{psec}{Trigonometric rules}


\end{psec}
