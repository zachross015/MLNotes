\documentclass{book}[a5paper]
\usepackage[a5paper]{geometry}
\usepackage[utf8]{inputenc}
\usepackage{amsmath}
\usepackage{amsthm}
\usepackage{amssymb}
\usepackage[font=small,labelfont=bf]{caption}

\usepackage{xfp}
\usepackage{pgfplots, pgfplotstable}
\usetikzlibrary{pgfplots.units}
\pgfplotsset{compat=1.16}

\usepackage{tikz}
\usepackage{float}

% Speed up the compile time for TIKZ
%\usepgfplotslibrary{external}
%\tikzexternalize

\newcommand{\placeholder}{[Insert Section Here]}

\newtheorem{theorem}{Theorem}[section]
\newtheorem{corollary}{Corollary}[theorem]
\newtheorem{lemma}[theorem]{Lemma}
\newtheorem*{remark}{Remark}

% Style to select only points from #1 to #2 (inclusive)
\pgfplotsset{select coords between index/.style 2 args={
    x filter/.code={
        \ifnum\coordindex<#1\def\pgfmathresult{}\fi
        \ifnum\coordindex>#2\def\pgfmathresult{}\fi
    }
}}

\title{Linear Regression}
\author{Zachary Ross}
\date{November 2019}

\begin{document}

\section{Real Value Prediction}

For this section, our motivation is to construct an algorithm whose focus is
predicting a value. That is, we want a \emph{prediction function} which takes an
input, called a \emph{feature}, and produces some numeric reponse, called a
\emph{scalar response}, between $-\infty$ and $\infty$. As with the rest of
machine learning and with life itself, we \emph{learn} using experiences, or
features.

% TODO: Include information on application

\subsection{Linear Regression}

% TODO: Site sources:
% - https://www.kaggle.com/c/house-prices-advanced-regression-techniques/data

The main focus of this section is determining a real number
that is associated with some other number. In a linear sense, we are finding
a continuous function that takes in one number and spits out another using
previously gathered data. So the question is, how do we form this function?


\begin{figure}[t!]
    \begin{tikzpicture}
        \selectcolormodel{gray}
        \begin{axis}[
                title = {HOUSE PRICES CORRELATED WITH PLOT AREA},  % whatever name you want
                xlabel = {Plot Area in 1,000 ft$^2$},
                ylabel = {House Price in \$100,000},
                tick label style={/pgf/number format/fixed},
                scaled y ticks = false,
                scaled x ticks = false,
                yticklabel={
                    \fpeval{\tick / 100000}               % Divide the y coordinate/1000
                },
                xticklabel={
                    \fpeval{\tick / 1000}               % Divide the y coordinate/1000
                },
            ]
            \addplot[
                only marks,
                select coords between index={1}{50},
                ] table[col sep=comma]{linreg_houseprice.csv};
        \end{axis}
    \end{tikzpicture}
    \caption{List of plot areas and selling prices for houses in Ames, Iowa.
    Looking at these kind of plots, we can try to find correlations in the data
    that help us predict what future houses may cost in that market.}
    \label{fig:hp}
\end{figure}

The scatterplot in Figure \ref{fig:hp} shows a bit of housing data from
Ames, Iowa. There is a positive correlation when comparing each plot area with its corresponding price, showing us that as plot area increases, so does the house price. We can ascribe this relationship to a variable $\theta_1$. That is, $\theta_1$ is the price per square foot of plot area ($\frac{\text{price}}{\text{ft}^2}$). Additionally, we can establish a \emph{base price} to this correlation. That is, we can instantiate some other variable, call it $\theta_0$, which acts as an offset for the housing prices. In context, this offset can be considered the starting price for houses in Ames, such that any plot will cost the base price of \$100,000 plus the price per square foot.

Examining these properties sets us up to form a hypothesis for how house prices
are determined.  If we let $\theta = \begin{pmatrix}\theta_0 \\ \theta_1\end{pmatrix}$ be a vector of the associated prices
with the base price as $\theta_0$ and price per ft$^2$ as $\theta_1$, and let $X^{(i)}$
be the plot area for the $i$th feature in the training set $X$, the equation
\begin{equation}
    hypothesis(\theta, X^{(i)}) = \theta_0 + \theta_1X^{(i)}
\end{equation}
should ideally give us the price for $X^{(i)}$. The key word in the last sentence is \emph{ideally} since the hypothesis function will only give us 100\% accuracy when all the data in the data set lies exactly on the line formed by the hypothesis. This would mean that each square foot of plot area would cost the exact same amount at every house you check. However optimal this is, it is generally not the case. Instead, we look to maximize the accuracy of this function or otherwise minimize the \emph{error}. We do this by altering the prices in $\theta$.

We begin measuring the error of our hypothesis by comparing it with
the previously collected data.  An intuitive way for measuring the  error of the hypothesis would be by measuring the shortest distance between the value it predicts and the recorded value since this, in a literal sense, tells us how far off our prediction was from the result. Let $y^{(i)}$ be the recorded house price
associated with a plot area $X^{(i)}$. This shortest distance would be the euclidean distance between the two, measured by:
\begin{equation}
    error(\theta, X^{(i)}, y^{(i)}) = (hypothesis(\theta, X^{(i)}) - y^{(i)})^2
\end{equation}
The will give us the error, or distance, between our prediction and our actual
results.

% TODO: Add graphs showing intuitive definition of error

The total error of the hypothesis will then be the total of all the errors between the training set predictions and the actual training set results. Let $m$ be the number of features in the training set $X$. Then
\begin{equation}
totalError(\theta, X, y) = \sum_{i=1}^m error(\theta, X^{(i)}, y^{(i)})
\end{equation}
will give us a good estimate for how well the weights $\theta$ are performing in the hypothesis.

\end{document}
