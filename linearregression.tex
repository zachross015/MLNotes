\documentclass{book}[a5paper]
\usepackage[a5paper]{geometry}
\usepackage[utf8]{inputenc}
\usepackage{amsmath}
\usepackage{amsthm}
\usepackage{amssymb}
\usepackage[font=small,labelfont=bf]{caption}

\usepackage{xfp}
\usepackage{pgfplots, pgfplotstable}
\usetikzlibrary{pgfplots.units}
\pgfplotsset{compat=1.16}

\usepackage{tikz}
\usepackage{float}

% Speed up the compile time for TIKZ
%\usepgfplotslibrary{external}
%\tikzexternalize

\newcommand{\placeholder}{[Insert Section Here]}

\newtheorem{theorem}{Theorem}[section]
\newtheorem{corollary}{Corollary}[theorem]
\newtheorem{lemma}[theorem]{Lemma}
\newtheorem*{remark}{Remark}

% Style to select only points from #1 to #2 (inclusive)
\pgfplotsset{select coords between index/.style 2 args={
    x filter/.code={
        \ifnum\coordindex<#1\def\pgfmathresult{}\fi
        \ifnum\coordindex>#2\def\pgfmathresult{}\fi
    }
}}

\title{Linear Regression}
\author{Zachary Ross}
\date{November 2019}

\begin{document}

\section{Real Value Prediction}

For this section, our motivation is to construct an algorithm whose focus is
predicting a value. That is, we want a \emph{prediction function} which takes an
input, called a \emph{feature}, and produces some numeric reponse, called a
\emph{scalar response}, between $-\infty$ and $\infty$. As with the rest of
machine learning and with life itself, we \emph{learn} using experiences, or
features.

% TODO: Include information on application

\subsection{Linear Regression}

% TODO: Site sources:
% - https://www.kaggle.com/c/house-prices-advanced-regression-techniques/data

The main focus of this section is determining a real number
that is associated with some other number. In a linear sense, we are finding
a continuous function that takes in one number and spits out another using
previously gathered data. So the question is, how do we form this function?


\begin{figure}[t!]
    \begin{tikzpicture}
        \selectcolormodel{gray}
        \begin{axis}[
                title = {HOUSE PRICES CORRELATED WITH PLOT AREA},  % whatever name you want
                xlabel = {Plot Area in 1,000$ft^2$},
                ylabel = {House Price in \$100,000},
                tick label style={/pgf/number format/fixed},
                scaled y ticks = false,
                scaled x ticks = false,
                yticklabel={
                    \fpeval{\tick / 100000}               % Divide the y coordinate/1000
                },
                xticklabel={
                    \fpeval{\tick / 1000}               % Divide the y coordinate/1000
                },
            ]
            \addplot[
                only marks,
                select coords between index={1}{50},
                ] table[col sep=comma]{linreg_houseprice.csv};
        \end{axis}
    \end{tikzpicture}
    \caption{List of plot areas and selling prices for houses in Ames, Iowa.
    Looking at these kind of plots, we can try to find correlations in the data
    that help us predict what future houses may cost in that market.}
    \label{fig:hp}
\end{figure}

Examine Figure \ref{fig:hp}, which shows a bit of housing data from
Ames, Iowa. Looking at the data, there is a positive correlation when comparing each plot area with its corresponding price. This correlation shows us that a plot area has a relationship with the house price, which we can ascribe to a variable $\theta_1$. That is, $\theta_1$ is the price per square foot of plot area ($\frac{\text{price}}{ft^2}$). Additionally and less obviously, we can establish a \emph{base price} to this correlation. That is, we can instantiate some other variable, call it $\theta_0$, which acts as an offset for the housing prices. In context, this would mean that $\theta_0$ acts as a starting point for the housing prices. So for the graph in \ref{fig:hp}, we may see a base price of \$100,000 which tells us that most houses in the data set are above this base price.

Examining these properties sets us up to form a hypothesis for how house prices
are determined.  If we let $\theta$ be a vector of the associated prices such that
$\theta_0$ is the base price and $\theta_1$ is the price per $ft^2$, and let $X$
be the plot area in $ft^2$, the equation
\begin{equation}
    hypothesis(\theta, X) = \theta_0 + \theta_1X
\end{equation}
should ideally give us the price for that plot area. This is the standard slope
equation $y=mx+b$, so this $hypothesis$ function will give us a straight line if
we fix the values of the associated prices $\theta$ to constants. Then the goal
for determining our hypothesis function is to \emph{find fixed values of
$\theta$ that will give us the most accurate price for a plot area $X$}.

So the next question is: how is accuracy determined? For this, we begin using
the previously collected data. Let $y$ be the recorded house price
associated with a plot area $X$. Then we measure the error of the hypothesis
function by the distance between the value it predicts and the recorded value
that was observed. That is
\begin{equation}
    error(\theta, X, y) = (hypothesis(\theta, X) - y)^2
\end{equation}
will give us the error, or distance, between our prediction and our actual
results.

Now that we can determine the accuracy, we want to find the values of $\theta$
that minimize this. Before we find the way to minimize this, let's make a couple
of observations first.


This happens when, for every feature $X^{(i)}$ in the data
set, $error(\theta, X^{(i)}, y)$ is as close to 0 as possible.
\end{document}
