\section{Gradient Descent}
\label{st:gradient descent}



The coordinates we are looking for are then the exact point where this
paraboloid bottoms out since this is by definition the minimum of the cost
function. We can obtain this point by noting the graph has an \emph{inflection
point} at the point $P = (\theta_x, \theta_y)$ where we go from descending the
paraboloid's curve to ascending it. That is, if we were to take points $P_1$,
$P_2$, $P_3$, and $P_4$ with $P_1 < P_2 < P < P_3 < P_4$ then $cost(P_2) -
cost(P_1)$ would be negative while $cost(P_4) - cost(P_3)$ would be positive. As
we approach $P$ from either side of the inflection, we expect to see the net
change being reduced to infinitely small amounts. This makes sense since at $P$
we are neither descending or ascending and as we move closer to it we see the
amount change reflecting this. This is by definition when the gradient of the
cost function is 0: $\nabla cost(\vec\theta) = \vec{0}$.

% TODO: Site source
% https://stanford.edu/~rezab/classes/cme323/S16/notes/Lecture03/cme323_lec3.pdf
Solving for this gradient is, however, computationally infeasible. As we will
see later on, the growth of this type of function is much too large for many
modern computers to handle efficiently, especially as we begin adding more
components to our features. Even for this simple example, we end up attempting
to find solutions for a linear system of the form $A\vec\theta = B$, and so we
could find these weights by computing $\vec\theta =A^{-1}B$. This has a lower
computational bound of $O(n^{2.375})$ when using a variant of \emph{Strassen's
algorithm} discovered by Virginia Williams [Source]. $n$ in this case is the
number of weights in $\vec\theta$. However, the algorithm tends to only be
advantageous over the naive version of matrix multiplication when a specific RAM
model (PRAM, for more see [source]) is used and when $n \geq 1,000$. Otherwise,
the traditional form of matrix multiplication is nearly as effective. This puts
us at a lower bound of $O(n^3)$, which is highly inefficient when dealing with
models of any caliber.
